% Options for packages loaded elsewhere
\PassOptionsToPackage{unicode}{hyperref}
\PassOptionsToPackage{hyphens}{url}
%
\documentclass[
]{article}
\usepackage{amsmath,amssymb}
\usepackage{iftex}
\ifPDFTeX
  \usepackage[T1]{fontenc}
  \usepackage[utf8]{inputenc}
  \usepackage{textcomp} % provide euro and other symbols
\else % if luatex or xetex
  \usepackage{unicode-math} % this also loads fontspec
  \defaultfontfeatures{Scale=MatchLowercase}
  \defaultfontfeatures[\rmfamily]{Ligatures=TeX,Scale=1}
\fi
\usepackage{lmodern}
\ifPDFTeX\else
  % xetex/luatex font selection
\fi
% Use upquote if available, for straight quotes in verbatim environments
\IfFileExists{upquote.sty}{\usepackage{upquote}}{}
\IfFileExists{microtype.sty}{% use microtype if available
  \usepackage[]{microtype}
  \UseMicrotypeSet[protrusion]{basicmath} % disable protrusion for tt fonts
}{}
\makeatletter
\@ifundefined{KOMAClassName}{% if non-KOMA class
  \IfFileExists{parskip.sty}{%
    \usepackage{parskip}
  }{% else
    \setlength{\parindent}{0pt}
    \setlength{\parskip}{6pt plus 2pt minus 1pt}}
}{% if KOMA class
  \KOMAoptions{parskip=half}}
\makeatother
\usepackage{xcolor}
\usepackage[margin=1in]{geometry}
\usepackage{graphicx}
\makeatletter
\def\maxwidth{\ifdim\Gin@nat@width>\linewidth\linewidth\else\Gin@nat@width\fi}
\def\maxheight{\ifdim\Gin@nat@height>\textheight\textheight\else\Gin@nat@height\fi}
\makeatother
% Scale images if necessary, so that they will not overflow the page
% margins by default, and it is still possible to overwrite the defaults
% using explicit options in \includegraphics[width, height, ...]{}
\setkeys{Gin}{width=\maxwidth,height=\maxheight,keepaspectratio}
% Set default figure placement to htbp
\makeatletter
\def\fps@figure{htbp}
\makeatother
\setlength{\emergencystretch}{3em} % prevent overfull lines
\providecommand{\tightlist}{%
  \setlength{\itemsep}{0pt}\setlength{\parskip}{0pt}}
\setcounter{secnumdepth}{-\maxdimen} % remove section numbering
\ifLuaTeX
  \usepackage{selnolig}  % disable illegal ligatures
\fi
\IfFileExists{bookmark.sty}{\usepackage{bookmark}}{\usepackage{hyperref}}
\IfFileExists{xurl.sty}{\usepackage{xurl}}{} % add URL line breaks if available
\urlstyle{same}
\hypersetup{
  hidelinks,
  pdfcreator={LaTeX via pandoc}}

\author{}
\date{\vspace{-2.5em}}

\begin{document}

This form documents the artifacts associated with the article (i.e., the
data and code supporting the computational findings) and describes how
to reproduce the findings.

\section{Part 1: Data}\label{part-1-data}

\begin{itemize}
\tightlist
\item[$\square$]
  This paper does not involve analysis of external data (i.e., no data
  are used or the only data are generated by the authors via simulation
  in their code).
\end{itemize}

\begin{itemize}
\tightlist
\item[$\boxtimes$]
  I certify that the author(s) of the manuscript have legitimate access
  to and permission to use the data used in this manuscript.
\end{itemize}

\subsection{Abstract}\label{abstract}

The dataset consists of observed daily precipitation data in millimeters
(mm) from 125 monitoring stations in the Danube river basin (Europe) and
from 2229 monitoring stations in the Mississippi river basin (North
America) over the period from 1965 to 2020. The temperature covariate in
Celsius degree used to fit the model in this project was derived from
the ERA5-Land reanalysis data for the corresponding region we selected,
which is a global land-surface dataset with a high spatial resolution of
9km, and the projected temperature covariate is derived from climate
models outputs of the sixth Coupled Model Intercomparison Project
(CMIP6), namely AWI, MIROC, and MPI.

\subsection{Availability}\label{availability}

\begin{itemize}
\tightlist
\item[$\boxtimes$]
  Data \textbf{are} publicly available.
\item[$\square$]
  Data \textbf{cannot be made} publicly available.
\end{itemize}

If the data are publicly available, see the \emph{Publicly available
data} section. Otherwise, see the \emph{Non-publicly available data}
section, below.

\subsubsection{Publicly available data}\label{publicly-available-data}

\begin{itemize}
\item[$\boxtimes$]
  Data are available online at:
\item
  Climate model outputs and EAR5-Land Reanalysis data:
  \url{https://www.copernicus.eu/en/access-data}
\item
  Precipitation observations:
  \url{https://www.ncei.noaa.gov/products/land-based-station/global-historical-climatology-network-daily}
\item
  Watershed Boundary Dataset:
  \url{https://www.usgs.gov/national-hydrography/watershed-boundary-dataset}
\item
  The data files for plots are stored in Github repository:
  \url{https://github.com/PangChung/ExtremePrecip/}
\item[$\square$]
  Data are available as part of the paper's supplementary material.
\item[$\square$]
  Data are publicly available by request, following the process
  described here:
\item[$\boxtimes$]
  Data are or will be made available through some other mechanism,
  described here:
\item
  We have saved the data used in our marginal modeling and dependence
  modeling in the format of R data file (.RData), and we host those data
  on Github{[}\url{https://github.com/PangChung/ExtremePrecip}{]} under
  the folder ``data'\,', which are public accessible. Though the raw
  data of the EAR5-Land Reanalysis data from which we derived the
  temperature covariate in Celsius degree can be downloaded from the
  website \url{https://www.copernicus.eu/en/access-data}, this raw data
  will be provided upon email request as the size of it is hundreds of
  gigabytes and it will take great efforts for someone to download
  directly from the Copernicus website. The email address is
  peng{[}dot{]}zhong{[}at{]}unsw{[}dot{]}edu{[}dot{]}au.
\end{itemize}

\subsubsection{Non-publicly available
data}\label{non-publicly-available-data}

\subsection{Description}\label{description}

\subsubsection{File format(s)}\label{file-formats}

\begin{itemize}
\tightlist
\item[$\square$]
  CSV or other plain text.
\item[$\boxtimes$]
  Software-specific binary format (.Rda, Python pickle, etc.): .Rda
  (.RData)
\item[$\square$]
  Standardized binary format (e.g., netCDF, HDF5, etc.):
\item[$\square$]
  Other (please specify):
\end{itemize}

\subsubsection{Data dictionary}\label{data-dictionary}

\begin{itemize}
\tightlist
\item[$\boxtimes$]
  Provided by authors in the following file(s):
\item
  \texttt{data/precip.RData}: The precipitation data for the eight
  subregions, which contain \texttt{R} objects:

  \begin{itemize}
  \tightlist
  \item
    \texttt{precip}: Lists of lists, raw data of the precipitation in
    millimeters (mm) in 8 subregions. This is the response variable used
    in the marginal fit and dependence fit.
  \item
    \texttt{region.name}: a vector, names of the 8 subregions
  \item
    \texttt{region.id}: a vector, regional ID number that corresponding
    to the regional number in the shapefiles.
  \item
    \texttt{station}: a data frame, contains geographical information
    about the monitoring stations, where the precipitation data were
    recorded. The
    \texttt{X\textquotesingle{}\textquotesingle{}\ column\ is\ latitude\ degrees,\ the}Y'\,'
    column is the longitude in degrees, the
    \texttt{start\textquotesingle{}\textquotesingle{}\ column\ is\ the\ start\ measuring\ year,\ the}end'\,'
    is the last measuring year, the
    \texttt{elev\textquotesingle{}\textquotesingle{}\ is\ the\ elevation\ of\ the\ monitoring\ stations\ in\ meters,\ and\ the}group.id'\,'
    corresponds to the region.id for identifying each of the 8
    subregions.
  \item
    \texttt{START.date\ and\ END.date}: dates, between which the data
    were used in our analysis.
  \end{itemize}
\item
  \texttt{data/temperature.RData}: The derived temperature covariate
  over the period 1965--2020, which contains \texttt{R} objects:

  \begin{itemize}
  \tightlist
  \item
    \texttt{date.df}: data frame, contains the date between 1965--2020,
    and its corresponding season of the year.
  \item
    \texttt{temperature.covariate}: list of vectors, the derived
    temperature covariate (30 day moving averages) used in marginal
    modeling and dependence modeling.
  \item
    \texttt{temperature}: list of vectors, daily spatial temperature
    avarages in Celsius degrees, used only during data preprocessing.
  \item
    \texttt{loc\_df}: geographical information about the locations of
    the ERA5-Land temperature data, used only during data preprocessing.
  \end{itemize}
\item
  \texttt{data/temperature\_pred.RData}: The derived temperature
  covariate from the climate models over the period 2015--2100 under
  different shared socioeconomic pathways (SSP 2-4.5 or SSP 5-8.5),
  which contains \texttt{R} objects:

  \begin{itemize}
  \tightlist
  \item
    \texttt{date.245\ and\ date.585}: vector of dates, corresponding to
    the temperature covariate from the climate models.s
  \item
    \texttt{idx.models}: 5 climate models, we used the first, the 3rd,
    and the 4th for future projections, which are denoted by AWI, MIROC,
    and MPI.
  \item
    \texttt{temperature.245.avg\ and\ temperature.585.avg}: lists of
    list, each list contains the derived temperature covariate in
    Celsius degrees from one climate model for future projections before
    realign with the temperature covariate derived from the ERA5-Land
    data under SSP 2-4.5 (or SSP 5-8.5). SSP 2-4.5 is denoted by 245,
    and the same goes for SSP 5-8.5.\\
  \end{itemize}
\item
  \texttt{data/marginal\_fit\_quantiles.RData}: Transformed margins
  based on the marginal fit, which contains \texttt{R} objects:

  \begin{itemize}
  \tightlist
  \item
    \texttt{theretical.quantiles}: list of matrixs, each matrix
    corresponding to the fitted marginal data based on the marginal
    model fit for each subregion, this is only used to generate the
    QQplots in the manuscript.
  \end{itemize}
\item
  \texttt{data/dep.fit.boot.results3.RData}: Fitted results from the
  bootstrap scheme for the dependence model, which contains \texttt{R}
  objects:

  \begin{itemize}
  \tightlist
  \item
    \texttt{boot.result.df}: data frame, contains the 95\% confidence
    interval (e.g, the column low.shape, high.shape) based on the
    bootstrap for each season, each region and each risk functional.
  \item
    \texttt{boot.result.list}: lists of lists for each risk functional
    (first dim), each season (second dim) and each subregion (third
    dimension). In each list, it contains standard error of the three
    parameter estimates, the estimates using the whole data, and the 300
    nonparametric boostrap estimates, named by jack.\\
  \end{itemize}
\item
  \texttt{data/era5\_geoinfo.RData}: Shapefiles for the eight
  subregions, which contains \texttt{R} objects:

  \begin{itemize}
  \tightlist
  \item
    \texttt{shape1}: shape file of the US river-basins
  \item
    \texttt{shape3}: shape file of Danube river-basin
  \end{itemize}
\item
  \texttt{data/transformed\_coordinates.RData}: Transformed coordinates
  for the eight subregions, which transformed the latitude/longitude
  coordinate to the Eculidean coordinate and contains \texttt{R}
  objects:

  \begin{itemize}
  \tightlist
  \item
    \texttt{loc.trans.list}: list of matrix: transfomred coordinates for
    each of the 8 subregions in meters.
  \end{itemize}
\item[$\square$]
  Data file(s) is(are) self-describing (e.g., netCDF files)
\item[$\boxtimes$]
  Available at the following URL:
\item
  Github: \url{https://github.com/PangChung/ExtremePrecip/}
\end{itemize}

\subsubsection{Additional Information
(optional)}\label{additional-information-optional}

\section{Part 2: Code}\label{part-2-code}

\subsection{Abstract}\label{abstract-1}

The code contains all the code to process the raw data (upon request)
and generate all the figures and tables in the main manuscript and the
supplemental material from the .RData files provided. The \texttt{R}
script file \texttt{code/bootstrap.R} is used to fit the marginal model
as well as the dependence model when the variable
\texttt{bootstrap.ind=0}, otherwise, it will fit the marginal model
together with the dependence model to the bootstrap data when the
variable \texttt{bootstrap.ind\ !=0}. To generate all the figures, one
can use and follow the \texttt{R} script \texttt{code/plots.R}. We also
created a \texttt{R} markdown file \texttt{ResultsReproduction.Rmd} to
help readers learn how to reproduce the results presented in this
manuscript.

\subsection{Description}\label{description-1}

\subsubsection{Code format(s)}\label{code-formats}

\begin{itemize}
\tightlist
\item[$\boxtimes$]
  Script files

  \begin{itemize}
  \tightlist
  \item[$\boxtimes$]
    R
  \item[$\square$]
    Python
  \item[$\square$]
    Matlab
  \item[$\square$]
    Other:
  \end{itemize}
\item[$\boxtimes$]
  Package

  \begin{itemize}
  \tightlist
  \item[$\boxtimes$]
    R
  \item[$\square$]
    Python
  \item[$\square$]
    MATLAB toolbox
  \item[$\square$]
    Other:
  \end{itemize}
\item[$\square$]
  Reproducible report

  \begin{itemize}
  \tightlist
  \item[$\square$]
    R Markdown
  \item[$\square$]
    Jupyter notebook
  \item[$\square$]
    Other:
  \end{itemize}
\item[$\boxtimes$]
  Shell script
\item[$\square$]
  Other (please specify):
\end{itemize}

\subsubsection{Supporting software
requirements}\label{supporting-software-requirements}

\begin{itemize}
\tightlist
\item
  de Fondeville R, Belzile L (2023). \emph{mvPot: Multivariate
  Peaks-over-Threshold Modelling for Spatial Extreme Events}. \texttt{R}
  package version 0.1.6,\url{https://CRAN.R-project.org/package=mvPot}.
\item
  Youngman BD (2022). ``evgam: An R Package for Generalized Additive
  Extreme Value Models.'' \emph{Journal of Statistical Software},
  \emph{103}(3), 1-26. \url{doi:10.18637/jss.v103.i03}
  \url{https://doi.org/10.18637/jss.v103.i03}.
\end{itemize}

\paragraph{Version of primary software
used}\label{version-of-primary-software-used}

\begin{itemize}
\tightlist
\item
  \texttt{R} version 4.3.2
\item
  \texttt{OpenPBS} 23.06.06
\end{itemize}

\paragraph{Libraries and dependencies used by the
code}\label{libraries-and-dependencies-used-by-the-code}

\begin{itemize}
\tightlist
\item
  \texttt{R} packages:

  \begin{itemize}
  \tightlist
  \item
    \texttt{mvPotST} (a spatial-temporal version of \texttt{mvPot}
    version 0.1.6,\\
    located at \texttt{code/archived/mvPotST/mvPotST\_0.0.0.9000.tar.gz}
    in the Github repository.)
  \item
    \texttt{evgam} version 1.0.0
  \item
    \texttt{mgcv} version 1.9-0
  \item
    \texttt{evd} version 2.3-6.1\\
  \item
    \texttt{lubridate} version 1.9.3
  \item
    \texttt{ggplot2} version 3.5.0
  \item
    \texttt{ggpubr} version 0.6.0
  \item
    \texttt{gridExtra} version 2.3
  \end{itemize}
\end{itemize}

\subsubsection{Supporting system/hardware requirements
(optional)}\label{supporting-systemhardware-requirements-optional}

Access to HPC with OpenPBS management software.

\subsubsection{Parallelization used}\label{parallelization-used}

\begin{itemize}
\tightlist
\item[$\square$]
  No parallel code used
\item[$\square$]
  Multi-core parallelization on a single machine/node

  \begin{itemize}
  \tightlist
  \item
    Number of cores used:
  \end{itemize}
\item[$\boxtimes$]
  Multi-machine/multi-node parallelization

  \begin{itemize}
  \tightlist
  \item
    Number of nodes and cores used: 56 nodes with 16 cores on each nodes
  \end{itemize}
\end{itemize}

\subsubsection{License}\label{license}

\begin{itemize}
\tightlist
\item[$\boxtimes$]
  MIT License (default)
\item[$\square$]
  BSD
\item[$\square$]
  GPL v3.0
\item[$\square$]
  Creative Commons
\item[$\square$]
  Other: (please specify)
\end{itemize}

\subsubsection{Additional information
(optional)}\label{additional-information-optional-1}

\section{Part 3: Reproducibility
workflow}\label{part-3-reproducibility-workflow}

\subsection{Scope}\label{scope}

The provided workflow reproduces:

\begin{itemize}
\tightlist
\item[$\square$]
  Any numbers provided in text in the paper
\item[$\boxtimes$]
  The computational method(s) presented in the paper (i.e., code is
  provided that implements the method(s))
\item[$\boxtimes$]
  All tables and figures in the paper
\item[$\square$]
  Selected tables and figures in the paper, as explained and justified
  below:
\end{itemize}

\subsection{Workflow}\label{workflow}

\subsubsection{Location}\label{location}

The workflow is available:

\begin{itemize}
\tightlist
\item[$\square$]
  As part of the paper's supplementary material.
\item[$\boxtimes$]
  In this Git repository:
  \url{https://github.com/PangChung/ExtremePrecip}
\item[$\square$]
  Other (please specify):
\end{itemize}

\subsubsection{Format(s)}\label{formats}

\begin{itemize}
\tightlist
\item[$\square$]
  Single master code file
\item[$\square$]
  Wrapper (shell) script(s)
\item[$\boxtimes$]
  Self-contained R Markdown file, Jupyter notebook, or other literate
  programming approach
\item[$\square$]
  Text file (e.g., a readme-style file) that documents workflow
\item[$\square$]
  Makefile
\item[$\square$]
  Other (more detail in \emph{Instructions} below)
\end{itemize}

\subsubsection{Instructions}\label{instructions}

Download the Github repository to your local machine, and see the
\texttt{R} markdown file \texttt{ResultsReproduction.Rmd} to learn how
to reproduce the results and all the figures presented in the
manuscript.

\subsubsection{Expected run-time}\label{expected-run-time}

Approximate time needed to reproduce the analyses on a standard desktop
machine:

\begin{itemize}
\tightlist
\item[$\square$]
  \textless{} 1 minute
\item[$\square$]
  1-10 minutes
\item[$\square$]
  10-60 minutes
\item[$\square$]
  1-8 hours
\item[$\square$]
  \textgreater{} 8 hours
\item[$\boxtimes$]
  Not feasible to run on a desktop machine, as described here: We use a
  computer cluster with 56 nodes to do the nonparametric bootstrap.
  However, one is still able to use a multicore workstation to fit the
  model for individual subregion within several hours.
\end{itemize}

\subsubsection{Additional information
(optional)}\label{additional-information-optional-2}

\section{Notes (optional)}\label{notes-optional}

\end{document}
